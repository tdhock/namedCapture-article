% !TeX root = RJwrapper.tex
\title{A new syntax for extracting tables of numbers from non-tabular text data using regular expressions}
\author{by Toby Dylan Hocking}

\maketitle

\abstract{Regular expressions are powerful tools for manipulating
  non-tabular textual data. For many tasks (visualization, machine
  learning, etc), tables of numbers must be extracted from such data
  before processing by other R functions. We present the R package
  namedCapture, which facilitates such tasks by providing a new
  user-friendly syntax for defining regular expressions in R code. We
  describe the history of regular expression support in R, and provide
  detailed comparisons with related R packages (rex, stringr, stringi,
  tidyr, rematch2, re2r).}

\section{Introduction}

Regular sets and regular expressions were introduced on paper by
Stephen Cole Kleene in 1956 (including the "Kleene star" * for zero
or more). Among the first uses of a regular expression in a program
was Ken Thompson (Bachelors 1965, Masters 1966, UC Berkeley) for his
version of the QED (1968) and ed (1969) text editors, developed at
Bell Labs for Unix. In ed, =g/re/p= means "Global Regular Expression
Print," which gave the name to the grep program, also written by
Thompson (1974). I'm not sure about the origin of capture groups, but
Friedl claimed that "The regular expressions supported by grep and
other early tools were quite limited...grep's capturing metacharacters
were =\(...\)=, with unescaped parentheses representing literal
text." Larry Wall wrote Perl version 1 in 1987 while working at Unisys
Corporation, and it had capturing regular expressions. Perl version 5
in 1994 introduced many extensions using the =(?= notation. Sources:
[[https://en.wikipedia.org/w/index.php?title%3DRegular_expression&oldid%3D682153405][wikipedia:Regular_expression]] and "A Casual Stroll Across the Regex
Landscape," in Ch.3 of Friedl's book Mastering Regular Expressions.

Philip Hazel started writing the Perl-Compatible Regular Expressions
(PCRE) library for the exim mail program in 1997. Python used PCRE
starting with version 1.5 in 1997. Source: [[file:python-1.5.2-Misc-HISTORY.txt][Python-1.5/Misc/HISTORY]].

#+BEGIN_SRC text
From 1.5a3 to 1.5a4...
- A completely new re.py module is provided (thanks to Andrew
Kuchling, Tim Peters and Jeffrey Ollie) which uses Philip Hazel's
"pcre" re compiler and engine.
#+END_SRC

Python 1.5 introduced named capture groups and the =(?P<name>subpattern)=
syntax. Source: [[file:python-1.5-Doc-libre.tex][Python-1.5/Doc/libre.tex]].

#+BEGIN_SRC tex
\item[\code{(?P<\var{name}>...)}] Similar to regular parentheses, but
the text matched by the group is accessible via the symbolic group
name \var{name}.
#+END_SRC

PCRE support for named capture was introduced in 2003. Source: [[http://pcre.org/original/changelog.txt][PCRE
changelog]] ([[file:pcre1-changelog.txt][my copy]]).

#+BEGIN_SRC text
Version 4.0 17-Feb-03...
36. Added support for named subpatterns. The Python syntax (?P<name>...) is
used to name a group. Names consist of alphanumerics and underscores, and must
be unique. Back references use the syntax (?P=name) and recursive calls use
(?P>name) which is a PCRE extension to the Python extension. Groups still have
numbers.
#+END_SRC

R includes PCRE starting with version 1.6.0 in 2002. Source:
[[file:R.NEWS.1.txt][R-src/NEWS.1]].

#+BEGIN_SRC text
CHANGES IN R VERSION 1.6.0...
    o	grep(), (g)sub() and regexpr() have a new argument `perl'
	which if TRUE uses Perl-style regexps from PCRE (if installed).
#+END_SRC

I wrote the code in https://svn.r-project.org/R/trunk/src/main/grep.c
which implements named capture regular expression support for R. It
was merged into base R in 2011
https://bugs.r-project.org/bugzilla3/show_bug.cgi?id=14518, and has
been included with every copy of R since version 2.14.

I wrote the =str.extractall= method in [[http://pandas.pydata.org/][pandas]], first included with
release version 0.18.0 ([[https://github.com/pydata/pandas/pull/11386][my Pull Request]] was merged in Feb 2016).


Introductory section which may include references in parentheses
\citep{R}, or cite a reference such as \citet{R} in the text.

\section{Another section}

This section may contain a figure such as Figure~\ref{figure:rlogo}.

\begin{figure}[htbp]
  \centering
  \includegraphics{Rlogo-5}
  \caption{The logo of R.}
  \label{figure:rlogo}
\end{figure}

\section{Another section}

There will likely be several sections, perhaps including code snippets, such as:

\begin{example}
  x <- 1:10
  result <- myFunction(x)
\end{example}

\section{Summary}

This file is only a basic article template. For full details of \emph{The R Journal} style and information on how to prepare your article for submission, see the \href{https://journal.r-project.org/share/author-guide.pdf}{Instructions for Authors}.

\bibliography{RJreferences}

\address{Author One\\
  Affiliation\\
  Address\\
  Country\\
  (ORCiD if desired)\\
  \email{author1@work}}

\address{Author Two\\
  Affiliation\\
  Address\\
  Country\\
  (ORCiD if desired)\\
  \email{author2@work}}

\address{Author Three\\
  Affiliation\\
  Address\\
  Country\\
  (ORCiD if desired)\\
  \email{author3@work}}
