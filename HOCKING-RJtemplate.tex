% !TeX root = RJwrapper.tex
\title{An efficient and readable syntax for defining named capture
  regular expressions in R code}

\author{by Toby Dylan Hocking}

\maketitle

\abstract{Regular expressions are powerful tools for manipulating
  non-tabular textual data. For many tasks (visualization, machine
  learning, etc), tables of numbers must be extracted from such data
  before processing by other R functions. We present the R package
  namedCapture, which facilitates such tasks by providing a new
  user-friendly syntax for defining regular expressions in R code. We
  begin by describing the history of regular expressions and their
  usage in R. We then describe the new features of the namedCapture
  package, and provide detailed comparisons with related R packages
  (rex, stringr, stringi, tidyr, rematch2, re2r).}
\section{Introduction}

Today regular expression libraries are powerful and widespread tools
for text processing. We begin by providing a brief history of regular
expressions and their usage in \R. We then provide an overview of
current R packages for regular expressions.

\subsection{Origin of regular expressions and named capture groups}

Regular expressions were first proposed on paper
by \citet{Kleene56}. Among the first uses of a regular expression in
computers was for searching in a text editor \citep{Thompson68} and
lexical processing of source code \citep{Johnson68}. 

A capture group in a regular expression is used to extract text that
matches a sub-pattern. In 1974, Thompson wrote the \texttt{grep}
command line program, which was among the first to support capture
groups \citep{Friedl2002}. In that program, backslash-escaped
parentheses \verb|\(\)| were used to open and close each capture
group, which could then be referenced by number (\verb|\1| for the
first capture group, etc).

The idea for named capture groups seems to have originated in 1994
with the contributions of Tracy Tims to Python 1.0.0, which used the
\verb|\(<labelname>...\)| syntax
\citep{Python-1.5.2-Misc-HISTORY}. Python 1.5 introduced the
\verb|(?P<labelname>...)| syntax for name capture groups
\citep{python-1.5-Doc-libre.tex}; the P was used to indicate that the
syntax was a Python extension to the standard.

Perl-Compatible Regular Expressions (PCRE) is a C library that is now
a widely used in free/open-source software tools such as Python and R.
PCRE introduced support for named capture in 2003, based on the Python
syntax \citep{pcre1-changelog.txt}. Starting in 2006, it supported the
\verb|(?<labelname>...)| and \verb|(?'labelname'...)| syntax, to be
consistent with Perl and .NET \citep{pcre1-changelog.txt}.

PCRE was first included in R version 1.6.0 in 2002
\citep{R.NEWS.1.txt}. The R functions \verb|regexpr| and
\verb|gregexpr| could be given the \verb|perl=TRUE| argument in order
to use the PCRE library. I wrote the C code that uses PCRE to extract
text matched by named capture groups \citep{HockingBug2011}, which was
accepted into R starting with version 2.14. I presented a lightning
talk at useR 2011 that showcased the new functionality
\citep{HockingUseR2011}.

\subsection{R packages for regular expressions}

Since the 

\citep{rex}.

\citep{stringr}

\citep{stringi}

\citep{tidyr}

\citep{rematch2}

\citep{re2r}

\citep{namedCapture}

\section{Another section}

This section may contain a figure such as Figure~\ref{figure:rlogo}.

\begin{figure}[htbp]
  \centering
  %\includegraphics{Rlogo-5}
  \caption{The logo of R.}
  \label{figure:rlogo}
\end{figure}

\section{Another section}

There will likely be several sections, perhaps including code snippets, such as:

\begin{example}
  x <- 1:10
  result <- myFunction(x)
\end{example}

\section{Summary}

This file is only a basic article template. For full details of \emph{The R Journal} style and information on how to prepare your article for submission, see the \href{https://journal.r-project.org/share/author-guide.pdf}{Instructions for Authors}.

\bibliography{RJreferences}

\address{Author One\\
  Affiliation\\
  Address\\
  Country\\
  (ORCiD if desired)\\
  \email{author1@work}}

\address{Author Two\\
  Affiliation\\
  Address\\
  Country\\
  (ORCiD if desired)\\
  \email{author2@work}}

\address{Author Three\\
  Affiliation\\
  Address\\
  Country\\
  (ORCiD if desired)\\
  \email{author3@work}}
