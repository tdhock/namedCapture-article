% !TeX root = RJwrapper.tex
\title{Capitalized Title Here}
\author{by Author One, Author Two and Author Three}

\maketitle

\abstract{Regular expressions are powerful tools for manipulating
  non-tabular textual data. For many tasks (visualization, machine
  learning, etc), tables of numbers must be extracted from such data
  before processing by other R functions. We present the R package
  namedCapture, which facilitates such tasks by providing a new
  user-friendly syntax for defining regular expressions in R code. We
  describe the history of regular expression support in R, and provide
  detailed comparisons with related R packages (rex, stringr, stringi,
  tidyr, rematch2, re2r).}

\section{Section title in sentence case}

Introductory section which may include references in parentheses
\citep{R}, or cite a reference such as \citet{R} in the text.

\section{Another section}

This section may contain a figure such as Figure~\ref{figure:rlogo}.

\begin{figure}[htbp]
  \centering
  \includegraphics{Rlogo-5}
  \caption{The logo of R.}
  \label{figure:rlogo}
\end{figure}

\section{Another section}

There will likely be several sections, perhaps including code snippets, such as:

\begin{example}
  x <- 1:10
  result <- myFunction(x)
\end{example}

\section{Summary}

This file is only a basic article template. For full details of \emph{The R Journal} style and information on how to prepare your article for submission, see the \href{https://journal.r-project.org/share/author-guide.pdf}{Instructions for Authors}.

\bibliography{RJreferences}

\address{Author One\\
  Affiliation\\
  Address\\
  Country\\
  (ORCiD if desired)\\
  \email{author1@work}}

\address{Author Two\\
  Affiliation\\
  Address\\
  Country\\
  (ORCiD if desired)\\
  \email{author2@work}}

\address{Author Three\\
  Affiliation\\
  Address\\
  Country\\
  (ORCiD if desired)\\
  \email{author3@work}}
